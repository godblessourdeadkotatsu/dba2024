\documentclass[class=article,crop=false]{standalone}
\usepackage{pacco}
\begin{document}
\subsection{A possible step further: Iterative Fast Fourier Transform Algorithm}
If we could arrange from the start the input components in a order that resembles the one reached by the deepest level of recursion (that is, when $N=1$), we could create a non-recursive approach for the FFT algorithm that works "bottom up". The idea behind this algorithm is the following:
\begin{enumerate}
\item we store in an array $A[0,\ldots,n-1]$ in the order in which they appear in the leaves of the recursion tree;
\item we take the elements in pairs, applying the butterfly operation that multiplies/adds two inputs with the twiddle factor to obtain $X_k=\text{evenFFT}_k+e^{-i\frac{2\pi}{N}k}\cdot \text{oddFFT}_k$ and $X_{k+\frac{N}{2}}=\text{evenFFT}_k-e^{-i\frac{2\pi}{N}k}\cdot \text{oddFFT}_k$. This will provide us with $N/2$ size 2 DFTs of each pair;
\item we replace the pairs in the array with their DFTs;
\item we take these $N/2$ DFTs (of size $2$ each) in pair and we compute the DFT of the $N/4$ groups of 4 input each that sit at the upper level in the recursion tree. This operation will apply two butterfly operations (since we are dealing with inputs of length 4);
\item we write these $N/4$ size 4 DFTs in the array;
\item we iterate the procedure untile the array holds 2 size $N/2$ DFTs, which we can combine using $N/2$ butterfly operations to obtain out final $N$ size DFT.
\end{enumerate}
To do this, we must firstly fill an array wtih the input component in the particular order that they appear in the leaves of the recursion graph. Let's examine the ordere in which the input components end up.\begin{center}
\begin{tabular}{|c| c c c c c c c c|}
    \hline
Original index &0 &1 &2 &3 &4 &4 &6 &7\\
\hline
Final index &0 &4 &2 &6 &1 &5 &2 &7\\
\hline
Original index (binary) &000 &001 &010 &011 &100 &101 &110 &111\\
\hline
Final index (binary) &000 &100 &010 &110 &001 &101 &110 &111\\
\hline
\end{tabular}\end{center}
We can see that the final index is the \textit{bit-reversed}\footnote{\cite{introalgo} p. 913} version of the original index: that is, to obtain the final index we must reverse the order of the bits that express the original index.
\begin{py}
def reverse(num, size):
result = 0
for _ in range(size): #shift result to the left bitwise and add the least significant bit to result
result = (result << 1) + (num & 1)
num >>= 1 #shift original number to the right bitwise (until none bits are left)
return result

def bit_reversed_order(lst):
n = len(lst)
#bit reversal depends on the size of the integer: for each index we need a number of bits equal to the number of bits used to represent the highest index
bit_size = n.bit_length() - 1 #using int class bit_length method

#note that the previous method for calculating the bit size only works because we expect the input to be a power of 2 and in binary powers of 2 are the numbers for which the number of bit needed for representation increases by 1. 
# If our input wasn't strictly in powers of 2 this method wouldn't be correct (for example, if n=5 then we would get a bit size of 2, which are too few to represent the number 4).

res = [0 for i in range(n)]
for i in range(n):
reversed_index = reverse(i, bit_size)
res[reversed_index] = lst[i]

return res
\end{py}

\end{document}

