% !TeX spellcheck = en_US
\documentclass[class=article,crop=false]{standalone}
\usepackage{pacco}
\begin{document}
\section{Introduction}
A classic problem in digital audio processing is the one of changing the tone and the speed of an audio file independently. 

Two intuitive examples of why speed and pitch are inherently linked can be found in the common experience with old physical music supports. Think about what happens when we set the speed of the turntable at 45 revolutions per minute while playing a vinyl record intended to be spun at 33 revolutions per minute: not only does the music speed up, but the whole sound of the record gets higher and squeakier (the so-called "chipmunk" effect). Or, on the converse, think about a cassette tape player with a faulty motor whose speed flutters, giving the impression that the music gets deeper when the tape slows down and after that goes back to normality when the speed of the reels picks up again. In both cases, the connection is quite direct: the reader head (the needle or the magnetic head) receives physical or magnetic stimulation at a slower or faster rate than intended, causing the modulation of the current to happen at a lower or higher frequency respectively. 

In digital audio the problem is still present: here audio information is recorded as a stream of samples ranging from 0 to 1 on a quantized scale (called \textit{bit depth}, typically 16 or 24 bit). These samples represents measurements of the variation in air pressure and are typically measured by a ADC (analog-digital converter) that takes an electrical signal (e.g. the electrical signal produced by a microphone) and samples it with a certain frequency (called \textit{sample rate}, typically 44.1 kHz or 48 kHz). These samples can be converted back to electrical signal via a DAC (digital-analog converter) to be monitored via analog devices (such as loudspeakers). 

The sample rate is now the analogue of the speed of the plate and the reels: if an audio file has a sample rate of 48 kHz it must be played back in such a way that the DAC processes 48000 samples every second. If we play back the file with a different playback rate, say 96 kHz, the time between the samples will be halved and the waveform of the audio file will "oscillate" twice as fast, doubling the pitch of the audio file (that is: playing it an octave higher). If we want to change playback rate without altering the audio signal we must perform a \textit{resampling} by changing the time relation between the samples. The difference with respect to analog supports lies in the Nyquist-Shannon sampling theorem, which states that a sampled signal can only represent frequencies up to $\frac{1}{2}$ its sample rate. This means that downsampling a signal will also "low-pass" it. \par 
But what if we wanted, for example, to slow down a recorded speech to better understand the words without the pitch dropping and rendering it hard to understand? Or, again, what if we want to adjust the vocal performance of an off-key singer without changing the duration of the notes?  And what if we are interested in changing the tempo of an audio file without affecting the pitch? This operation, known as \textit{time stretching}, can be also seen as a simple consequence of pitch shifting: if we want to double the length of an audio file we could simply pitch-shift it to a double pitch and the play it back at half the rate, bringing the pitch back to the initial one but simultaneously doubling the length.

So, we are left with the problem of devising a method to change the pitch of an audio signal without affecting its duration. This is a well-established problem in digital signal processing and digital audio and many different algorithms. Two main approaches are usually distinguished:
\begin{itemize}
	\item the \enf{time-domain} approach: the audio information is sliced and manipulated by working on the audio samples directly. Typical time-domain algorithms for pitch shifting include\footnote{\cite{dafx}, sec. 6.4} using a delay line, the OverLap-Add (OLA) method\footcite[][27]{theo} and the Pitch-Syncronous OverLap-Add (PSOLA);
	\item the \enf{frequency-domain} approach: the audio information is transformed from the time-domain (that is, as a function of time) into the frequency domain (that is, as a function of different frequency channels or bins). The manipulation then happens at the frequency level before transforming the audio information back to the time domain. In this field the most prominent technique is the \textit{phase vocoder}.
\end{itemize}
These approaches have both advantages and drawback, but most of the modern pitch shifting techniques employ the frequency domain approach, which is what this brief discussion will be focused on. We will start analyzing the most important mathematical operation at the base of any frequency-domain analysis, the discrete Fourier transform, and the efficient algorithm that contributed to shape the world of digital signal processing as we know it today: the fast Fourier transform.
\end{document}